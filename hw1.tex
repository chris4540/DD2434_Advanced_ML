% --------------------------------------------------------------
% This is all preamble stuff that you don't have to worry about.
% Head down to where it says "Start here"
% --------------------------------------------------------------
 
\documentclass[12pt]{article}
 
\usepackage[margin=1in]{geometry} 
\usepackage{amsmath,amsthm,amssymb}
\usepackage{mathtools}
\usepackage{graphicx}
\usepackage{tikz}
\usepackage{subfig}
 
\newcommand{\N}{\mathbb{N}}
\newcommand{\Z}{\mathbb{Z}}
 
\newenvironment{theorem}[2][Theorem]{\begin{trivlist}
\item[\hskip \labelsep {\bfseries #1}\hskip \labelsep {\bfseries #2.}]}{\end{trivlist}}
\newenvironment{lemma}[2][Lemma]{\begin{trivlist}
\item[\hskip \labelsep {\bfseries #1}\hskip \labelsep {\bfseries #2.}]}{\end{trivlist}}
\newenvironment{exercise}[2][Exercise]{\begin{trivlist}
\item[\hskip \labelsep {\bfseries #1}\hskip \labelsep {\bfseries #2.}]}{\end{trivlist}}
\newenvironment{reflection}[2][Reflection]{\begin{trivlist}
\item[\hskip \labelsep {\bfseries #1}\hskip \labelsep {\bfseries #2.}]}{\end{trivlist}}
\newenvironment{proposition}[2][Proposition]{\begin{trivlist}
\item[\hskip \labelsep {\bfseries #1}\hskip \labelsep {\bfseries #2.}]}{\end{trivlist}}
\newenvironment{corollary}[2][Corollary]{\begin{trivlist}
\item[\hskip \labelsep {\bfseries #1}\hskip \labelsep {\bfseries #2.}]}{\end{trivlist}}
\newenvironment{question}[2][Question]{\begin{trivlist}
\kern10pt
\item[\hskip \labelsep {\bfseries #1}\hskip \labelsep {\bfseries #2.}]}{\end{trivlist}}

\newcommand*{\answer}{%
  \par
  \kern1pt
  \begingroup
    \centering
    \raisebox{.2\baselineskip}{%
      \textcolor{gray}{
	    \rule{.6667\linewidth}{.1pt}%
      }
    }%
    \par
  \kern8pt
  \endgroup
}

\begin{document}
 
% --------------------------------------------------------------
%                         Start here
% --------------------------------------------------------------
 
%\renewcommand{\qedsymbol}{\filledbox}
 
\title{DD2434 Machine Learning, Advanced Course Assignment 1}
\author{Lin Chun Hung, chlin3@kth.se} 
 
\maketitle

% Question 1
\begin{question}{1}
Consider input output pairs are linked by the mapping to have the following
 relation:
\begin{equation}
    \boldsymbol{t}_i = f(\boldsymbol{x}_i) + \epsilon_i
\end{equation}
where the $\epsilon_i$ is the unbiased random noise. Since we have no piror knowledge
on the random noise term, the random noise is then assumed as following the normal
distribution. \\

With the assumption that features are uncorrelated, we choose the spherical
covariance matrix for the likelihood.
\end{question} % End question 1

% Question 2
\begin{question}{2}
Consider the general product rule of probability:
$$\mathrm {P} \left(\bigcap _{k=1}^{n}A_{k}\right)=
  \prod _{k=1}^{n}\mathrm {P} \left(A_{k}\,{\Bigg |}\,\bigcap _{j=1}^{k-1}A_{j}\right)$$

Therefore the likelihood would be:
\begin{equation}
  p(\boldsymbol{T}\mid f,\boldsymbol{X}) =
  \prod _{i=1}^{N}p(\boldsymbol{t}_i \mid \boldsymbol{t}_{i-1},...,\boldsymbol{t}_{1},
  f,\boldsymbol{X})
\end{equation}

\end{question} % End question 2

% --------------------------------------------------------------
%     You don't have to mess with anything below this line.
% --------------------------------------------------------------
 
\end{document}
